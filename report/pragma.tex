% Permission is granted to copy, distribute and/or modify this document
% under the terms of the GNU Free Documentation License, Version 1.3
% or any later version published by the Free Software Foundation;
% with no Invariant Sections, no Front-Cover Texts, and no Back-Cover Texts.
% A copy of the license is included in the section entitled "GNU
% Free Documentation License".
%
% Authors:
% Caner Candan <caner@candan.fr>, http://caner.candan.fr
% Aurèle Mahéo <aurele.maheo@gmail.com>

\section{Les directives pragma}

\emph{Qui n'a jamais révé de percer le mystère de ``OpenMP'' et de ses directives si joliment faites.}

\subsection{Modularité}

Comme introduit précédement les directives pragma font partie des fonctionnalités modulable. Il est très facile de créer une nouvelle directive dans le projet.

\subsection{Respect du sujet}

Il a été demandé dans le sujet de respecter deux styles de déclaration des pragmas que nous citons en figure \ref{fig:style1} et \ref{fig:style2}.

\begin{figure}[here]
  \centering
  \verb#pragma instrumente foo#
  \caption{Style 1}
  \label{fig:style1}
\end{figure}

\begin{figure}[here]
  \centering
  \verb#pragma instrumente (fct1, fct2)#
  \caption{Style 2}
  \label{fig:style2}
\end{figure}

\subsection{Fichiers de test}

Un ensemble de fichiers dans les répertoires ``test'' du projet ont été crées afin d'assurer le fonctionnement des fonctionnalités demandées.

\subsection{Gestion d'erreur}

Une gestion d'erreur est egalement garantie au cas où les styles cités précedement ne sont pas respectés. Cela se traduit par un message d'erreur du compilateur.
